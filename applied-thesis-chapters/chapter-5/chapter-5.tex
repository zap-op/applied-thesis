\chapter{TỔNG KẾT VÀ ĐÁNH GIÁ}

\section{Kiến thức thu được}

\begin{itemize}
    \item Trong quá trình thực hiện khóa luận, nhóm đã học tập, cải thiện, áp dụng được nhiều kĩ năng và kiến thức. Rèn luyện các kiến thức chuyên môn, kinh nghiệm làm việc và xử lý các vấn đề gặp phải về kĩ năng cứng lẫn kĩ năng mềm:
    \item Nâng cao kỹ năng trình bày báo cáo và viết tài liệu một cách chuyên nghiệp và khoa học đúng, đảm bảo các quy chuẩn được đưa ra.
    \item Củng cố, cập nhật kiến thức và áp dụng phương pháp quản lý công việc bằng mô hình Kanban, cơ hội được sử dụng công cụ Jira Software, Confluence chuyên sâu trong suốt quá trình thực hiện.
    \item Khả năng làm việc nhóm, công việc và vai trò giữa các thành viên được phân hóa rõ ràng hơn. Nâng cao hiệu suất công việc.
    \item Cải thiện khả năng tự học, tìm kiếm và đọc hiển tài liệu chuyên ngành, cũng như cải nâng cao khả năng hệ thống và triển khai kiến thức.
    \item Khả năng tự chủ, độc lập và tinh thần trách nhiệm trong công việc.
    \item Rèn luyện kỹ năng ước lượng, làm việc theo kế hoạch và hoàn thành công việc một cách hiệu quả.
    \item Tham gia phát triển từ đầu vòng đời của một phần mềm một cách chuyên nghiệp. Thử sức ở nhiều vai trò như thiết kế giao diện; thiết kế kiến trúc; phát triển và bảo trì phần mềm; điều hành và quản lý công việc; soạn thảo và cập nhật tài liệu liên quan.
    \item Thực hiện soạn thảo và cập nhật tài liên quan đến thiết kế, phát triển, kiểm thử, triển khai và hướng dẫn một cách chuyên môn.
    \item Nâng cao khả năng học hỏi công nghệ và kiến thức mới, kỹ năng viết mã nguồn, quản lý và khắc phục lỗi. Được trải qua quá trình đánh giá và chuyển đổi công nghệ.
    \item Tiếp cận và áp dụng thực tiễn các kiến thức về Virtual Private Server (VPS) một các chuyên sâu.
\end{itemize}

\section{Sản phẩm thu được}

\subsection{Môi trường phát triển}

\begin{itemize}
    \item 
\end{itemize}
Hệ điều hành: Windows 10 Home 64-bit, macOS Ventura

Hệ quản trị cơ sở dữ liệu: MongoDB

Công cụ quản lý, truy vấn cơ sở dữ liệu MongoDB: MongoDB Compass Community 1.8.0

Công cụ xây dựng và phát triển phần mềm: Visual Studio Code

Công cụ quản trị máy chủ từ xa qua giao thức SSH: Window PowerShell 5.1.19041.2673

Các thư viện / nền tảng sử dụng:

Tên thư viện / nền tảng

Tóm tắt chức năng

NodeJS

Là môi trường thực thi JavaScript bên ngoài trình duyệt, hỗ trợ viết mã phía server và xây dựng ứng dụng web có tính tương tác cao.

Express.js

Là framework Node.js hỗ trợ viết mã backend, cung cấp các tính năng như định tuyến, middleware và tương tác với database để xây dựng RESTful API và ứng dụng web.

Mongoose

Là thư viện Object Data Modeling (ODM) dành cho MongoDB, giúp đơn giản hóa truy vấn, tạo ra các model đồng thời cung cấp API để thao tác với database.

ReactJS

Là thư viện JavaScript được sử dụng để xây dựng giao diện, mang lại trải nghiệm người dùng tốt hơn qua các tính năng hiệu suất tối ưu, cập nhật dữ liệu, tái sử dụng Components và Virtual DOM.

Redux

Là thư viện quản lý trạng thái (state management) cho các ứng dụng web được xây dựng bằng ReactJS, giúp dễ dàng quản lý và cập nhật trạng thái của ứng dụng.

Bảng 5.1 Các thư viện, nền tảng sử dụng trong quá trình xây dựng hệ thống

5.2.2 Môi trường triển khai
 DigitalOcean - nhà cung cấp dịch vụ đám mây (cloud service provider), cung cấp các ứng dụng web chạy trên nền tảng máy chủ ảo và container, hỗ trợ phát triển và vận hành ứng dụng web nhanh chóng và tiết kiệm chi phí. Thông qua DigitalOcean, người dùng có thể thuê máy chủ ảo (VPS), quản lý hệ thống và cấu hình chi tiết.

Amazon Elastic Compute Cloud (EC2) - dịch vụ đám mây của Amazon Web Services (AWS), cung cấp máy chủ ảo (VPS) có thể được sử dụng để triển khai, quản lý và mở rộng các ứng dụng web một cách dễ dàng và tiết kiệm chi phí.

5.2.3 Hiệu năng hệ thống
5.2.4 Các chức năng đã cài đặt
Đăng ký / Đăng nhập.

Dùng thử chức năng quét với ZAP Spider.

Quét ứng dụng web qua URL hoặc IP bằng các loại quét, kịch bản nền tảng ZAP có hỗ trợ.

Hỗ trợ cấu hình các phiên quét ánh xạ với nền tảng ZAP.

Hỗ trợ quét Nmap Port.

Hỗ trợ quét Sslyze TLS/SSL Security.

Quản lý thông tin các URL hoặc IP dùng để quét và các thông tin liên quan.

Giao diện quá trình quét trực quan.

Quản lý thông tin kết quả của các phiên quét.

Gửi kết quả trạng thái quét, báo cáo rút gọn qua mail.

Xuất báo cáo với định dạng pdf, xml.

5.2.5 So sánh với một số hệ thống khác trên thị trường
Tên chức năng

Owlens

Stack Hawk

Detectify

Hosted Scan

Đăng ký / Đăng nhập.

:heavy_check_mark: 

:heavy_check_mark:  

:heavy_check_mark:  

:heavy_check_mark:  

Dùng thử chức năng quét với ZAP Spider.

:heavy_check_mark:  

:heavy_check_mark:  

 

:heavy_check_mark:  

Quét ứng dụng web qua URL hoặc IP bằng các loại quét, kịch bản nền tảng ZAP có hỗ trợ.

:heavy_check_mark:  

:heavy_check_mark:  

 

:heavy_check_mark:  

Hỗ trợ cấu hình các phiên quét ánh xạ với nền tảng ZAP.

:heavy_check_mark:  

:heavy_check_mark:  

 

 

Hỗ trợ quét Nmap Port.

:heavy_check_mark:  

:heavy_check_mark:  

:heavy_check_mark:  

:heavy_check_mark:  

Hỗ trợ quét Sslyze TLS/SSL Security.

:heavy_check_mark:  

:heavy_check_mark:  

:heavy_check_mark:  

 

Quản lý thông tin các URL hoặc IP dùng để quét và các thông tin liên quan.

:heavy_check_mark:  

:heavy_check_mark:  

:heavy_check_mark:  

:heavy_check_mark:  

Giao diện quá trình quét trực quan.

:heavy_check_mark:  

:heavy_check_mark:  

:heavy_check_mark:  

 

Quản lý thông tin kết quả của các phiên quét.

:heavy_check_mark:  

:heavy_check_mark:  

:heavy_check_mark:  

:heavy_check_mark:  

Gửi kết quả trạng thái quét, báo cáo rút gọn qua mail.

:heavy_check_mark:  

:heavy_check_mark:  

:heavy_check_mark:  

:heavy_check_mark:  

Xuất báo cáo với định dạng pdf, xml.

:heavy_check_mark:  

:heavy_check_mark:  

:heavy_check_mark:  

:heavy_check_mark:  

Bảng 5.2 So sánh chức năng giữa các hệ thống cung cấp dịch vụ

5.3 So sánh các kết quả thu được với mục tiêu ban đầu
Mục tiêu ban đầu

Nhận xét mức độ hoàn thành

Mục tiêu ban đầu

Nhận xét mức độ hoàn thành

Trình bày lý do xây dựng hệ thống tự động tìm kiếm lỗi bảo mật ứng dụng web dựa trên nền tảng OWASP ZAP.

Đã trình bày một số lý do cơ bản trong chương 1 của luận văn.

Trình bày các tính năng cơ bản của 3 ứng dụng tìm kiếm lỗi bảo mật ứng dụng web hiện có và nêu ra các khuyết điểm của các ứng dụng.

Đã trình bày một số ý ở chương 1 của luận văn.

Trình bày tóm tắt thông tin về các lỗi bảo mật ứng dụng web thường gặp.

Đã trình bày ở chương 2 của luận văn.

Xây dựng ứng dụng web tự động tìm kiếm lỗi bảo mật ứng dụng web hoàn thiện.

Đã xây dựng ứng dụng web tự động tìm kiếm lỗi bảo mật ứng dụng web.

Viết 120 trang luận văn theo luồng logic trình bày trong tài liệu “Hướng dẫn thực hiện luận văn” mà giáo viên cung cấp, theo đúng chuẩn nhà trường yêu cầu và trích dẫn tài liệu tham khảo một cách chi tiết, đầy đủ.

Luận văn được viết tương đối đẩy đủ và chính xác.

 Bảng 5.3 Bảng so sánh mục tiêu ban đầu với kết quả thu được