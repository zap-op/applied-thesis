\newpage
\phantomsection
\addcontentsline{toc}{chapter}{TÓM TẮT KHÓA LUẬN}
\begin{center}
\textbf{\Large TÓM TẮT KHÓA LUẬN }
\vspace{1cm}
\end{center}

Luận văn là tài liệu chính của khóa luận, nội dung của luận văn sẽ đề cập tới các kiến thức, kỹ thuật liên quan trong quá trình thực hiện khóa luận. Luận văn của nhóm em bao gồm các phần như sau:

Chương 1 - Giới thiệu đề tài: Trình bày lý do xây dựng hệ thống tự động tìm kiếm lỗi bảo mật ứng dụng web dựa trên nền tảng ZAP. Trình bày các tính năng cơ bản của 3 hệ thống tự động tìm kiếm lỗi bảo mật ứng dụng web tương tự hiện có. Đánh giá ưu điểm và các điểm chưa hoàn thiện của 3 hệ thống trên. Trình bày mục tiêu và phạm vi phát triển của đề tài.

Chương 2 - Lý thuyết nền tảng: Trình bày tóm tắt lý thuyết và thông tin về các lỗi bảo mật web phổ biến, kiến trúc phân tầng (N-Tier) và kiến trúc nguồn sự kiện (Event Sourcing). 

Chương 3 - Thiết kế giải pháp: Nếu ra các giải pháp chức năng cho ứng dụng hệ thống, thiết kế kiến trúc hệ thống, thiết kế giao diện hệ thống và các sơ đồ thiết kế liên quan.

Chương 4 - Cài đặt giải pháp: Đề cập các công cụ dùng để cài đặt từng giải pháp hoặc thành phần đã nói ở trong chương 3, đề cập cách hoạt động, cách cài đặt chi tiết của các chức năng. Bên cạnh đó, chương 4 còn đề cập đến một số khó khăn đáng kể trong quá trình cài đặt và giải pháp cho chúng.

Chương 5 - Tổng kết và đánh giá: Phần này liệt kê các kiến thức mà chúng em đạt được trong quá trình thực hiện luận văn, sản phẩm thu được từ quá trình thực hiện luận văn, những chức năng đã thực hiện. Cùng với đó, chương còn so sánh ứng dụng của luận văn với một số ứng dụng hiện có trên thị trường, phương hướng phát triển trong tương lai.